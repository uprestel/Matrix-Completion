\documentclass{article}
\usepackage{blindtext}
\usepackage[T1]{fontenc}
\usepackage[utf8]{inputenc}
\usepackage{amsmath}
\usepackage{amssymb}
\usepackage{algorithm}
\usepackage{algorithmicx}
\usepackage[noend]{algpseudocode}
\title{Compressed Sensing}
%\author{Ulrich P}
\date{\today}

\makeatletter
\def\BState{\State\hskip-\ALG@thistlm}
\makeatother



\numberwithin{equation}{section}
\DeclareMathOperator*{\argmax}{arg\,max}
\DeclareMathOperator*{\argmin}{arg\,min}

\DeclareMathOperator{\Prox}{Prox}
\DeclareMathOperator{\diag}{diag}


\begin{document}

\maketitle

\section{Preliminaries}

We define useful mappings used in convex optimization, namely the proximal mapping

$$\Prox_f^\lambda(y) = \argmin_x \Big( f(x) + \frac{1}{2 \lambda} || x - y|| \Big)$$

\section{Overview}
We are given an incomplete matrix $M$ with known entries $(i,j) \in \Omega$. In the general form matrix completion reads

$$\min ||X||_* \ \ \ \text{s.t.} \ \ \ \mathcal{A}(X) = B$$
%We need to find the min rank $X$ such that $AX = B$
where $\mathcal{A}$ is a linear map $\mathcal{A}:\mathbb{R}^{m \times n} \longrightarrow \mathbb{R}^{|\Omega|}$

and $B$ encodes our knowlege of $M$.
In our case we choose $ \mathcal{A} = \mathcal{P}_\Omega$ with
$$
[\mathcal{P}_\Omega(X)]_{i,j}= 
\begin{cases}
    X_{j+mi} & \text{if } (i,j) \in \Omega\\
    0,              & \text{otherwise}
\end{cases}
$$
We denote $A \in \mathbb{R}^{|\Omega| \times mn}$ the matrix corresponding to the mapping $\mathcal{P}_\Omega$. 
With this we can rewrite the objective above:
$$\min ||X||_* \ \ \text{s.t.} \ \ \ \mathcal{P}_\Omega(X) = A\text{vec}(X) = B = \mathcal{P}_\Omega(M)$$
\section{Douglas- Rachford Splitting}
In the noiseless case we can use DRS to solve this numerically via a fixed-point iteration: 
$$z^{(k+1)} = z^{(k)} + \Prox_{g}^\gamma\Big(2 \Prox_{f}^\gamma(z^{(k)}) - z^{(k)}\Big) - \Prox_{g}^\gamma(z^{(k)})$$
To do this, we express our objective as a sum of two functions $f + g$ where $f(X) = \delta_C(X)$ with $C = \lbrace X | AX = B \rbrace$ and $g(X) = ||X||_*$. We then obtain
\begin{align*}
\Prox_{f}^\gamma(x) = \Pi_C(x) &= x + A^+(b-Ax)
\\ &= x + A^T(A A^T)^{-1}(b - Ax)
\\ &= x + A^T(b-Ax)
\end{align*}
To express the proximal mapping of the nuclear norm, we define the singular value thresholding operator $\mathcal{D}_\gamma$. For a matrix $Y$ with the SVD
$Y = U \Sigma V^T = U \diag(\sigma(Y)) V^T$ it is expressed by $\mathcal{D}_\gamma (Y) = U \diag(\sigma(Y) - \lambda)_+ V^T$. Thus we get
\begin{align*}
\Prox_{g}^\gamma(x) &= U (\sigma(x) - \gamma)_{+} V^T
\\ &= \mathcal{D}_\gamma(x)
\end{align*}


\section{FISTA}

In the noisy case we are trying to minimize the following objective (in the Lagrangian formulation):

\begin{align}
\min_X \lambda||X||_*  + \frac{1}{2} ||AX - B||_2^2
\label{fista_objective}
\end{align}
For some $\lambda > 0$.
The FISTA algorithm is formulated with respect to the following objective:

$$\min_X f(x) + g(x)$$

where we assume $f$ and $g$ to be sufficiently smooth, i.e. $f \in C^{1,1}(\mathbb{R}^n)$, which means

$$\exists L(f) : ||\nabla f(x) - \nabla f(y)|| \leq L(f)||x - y|| \forall x,y \in \mathbb{R}^n$$

In our case we have $f(x) = \frac{1}{2} ||AX - B||_2^2 = \frac{1}{2}\langle AX - B, AX - B \rangle$ and $g(x) =  \lambda||X||_*$ .By simple calculation we get 

\begin{align}
\nabla f(X) &= A^TAX - A^TB = A^T(AX - B) \label{nabla_eq}
\\
L(f) &= 2\lambda_{\max}(A^TA)
\end{align}


\begin{algorithm}[H]
\caption{FISTA with constant step size}
\hspace*{\algorithmicindent} \textbf{Input} Lipschitz- constant $L(f)$ of $\nabla f$, $y_1 = x_0 \in \mathbb{R}^n, t_1=1$
\begin{algorithmic}[1]
\For{$k = 1, ...$}                    
	\State {$x_k = p_L(y_k)$}
	\State {$t_{k+1} = \frac{1 + \sqrt{1 + 4*t_k^2}}{2}$}
	\State {$y_{k+1} = x_k + \frac{t_k -1}{t_{k+1}}(x_k - x_{k-1})$}
\EndFor
\end{algorithmic}
\end{algorithm}


In this algorithm we def. 
\begin{align*}
Q_L(x,y) &= f(y) + \langle x-y, \nabla f(y) \rangle + \frac{L(f)}{2} ||x-y||^2 + g(x)\\
p_L(y) &= \argmin_x Q_L(x,y)\\
&= \argmin_x \Big( g(x) + \frac{L(f)}{2} \Big\| x - \Big( y - \frac{1}{L} \nabla f(y)\Big)\Big\|^2 \Big)\\
&= \argmin_x \Big( \frac{g(x)}{\lambda} + \frac{L(f)}{2\lambda} \Big\| x - \Big( y - \frac{1}{L} \nabla f(y)\Big)\Big\|^2 \Big)\\
&= \Prox_{g/\lambda}^{\lambda/L} \Big(y - \frac{1}{L}\nabla f(y)\Big)
\end{align*}
Since we are trying to find the proximal mapping for the nuclear norm, we can apply the thresholding operator just like in our iteration step for DRS. We get 
\begin{align*}
\Prox_{g/\lambda}^{\lambda/L}  \Big(y - \frac{1}{L} \nabla f(y) \Big) &= U \diag \Big(\sigma \Big(y - \frac{1}{L} \nabla f(y)\Big) - \frac{\lambda}{L}\Big)_+ V^T\\
&= \mathcal{D}_{\lambda / L}\Big(y - \frac{1}{L} \nabla f(y)\Big)\\
&= \mathcal{D}_{\lambda / L}\Big(y - \frac{1}{L} A^T(Ay - B)\Big)
\end{align*}

\textbf{I'm not sure if this is correct}
\end{document}